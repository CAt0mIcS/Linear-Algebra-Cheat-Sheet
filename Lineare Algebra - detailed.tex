\documentclass[german, fourcolumn, 8pt]{latex4ei/latex4ei_sheet}
% LastPage
\usepackage{lastpage}
\usepackage{amsmath}

% Allow hyperlinks
	\RequirePackage[pagebackref=true,pdfpagelabels]{hyperref}
	
% Colors
	\RequirePackage{latex4ei/latex4ei_colors}
	\colorlet{col_link}{tum_blue_dark}
	\hypersetup{
	colorlinks=true,
	linkcolor=col_link,
	urlcolor=col_link,
	citecolor=col_link,
}

\newcommand{\equival}[0]{\ensuremath{\Longleftrightarrow}}
% \newcommand{\spann}[1]{
%     \IfNoValueTF{#1}{\ensuremath{\operatorname{spann }}}{\ensuremath{\operatorname{spann(#1)}}}
% }
% \newcommand{\Rang}[1]{
%     \IfNoValueTF{#1}{\ensuremath{\operatorname{Rang }}}{\ensuremath{\operatorname{Rang(#1)}}}
% }
% \newcommand{\Kern}{\ensuremath{\operatorname{Kern }}}
% \newcommand{\Bild}{\ensuremath{\operatorname{Bild }}}
	
\DeclareMathOperator{\spann}{spann}
\DeclareMathOperator{\Rang}{Rang}
\DeclareMathOperator{\Kern}{Kern}
\DeclareMathOperator{\Bild}{Bild}
\DeclareMathOperator{\End}{End}
\DeclareMathOperator{\Iso}{Iso}
\DeclareMathOperator{\Aut}{Aut}

\begin{document}

\titlespacing*{\subsubsection} {0pt}{0.5ex}{0.0ex}
\titlespacing*{\subsection} {0pt}{1.0ex}{1.0ex}
\titleformat{\subsubsection}
{\normalfont\normalsize\bfseries}{\thesubsubsection}{.5em}{}


% Aufteilung in Spalten
\vspace{-4mm}
\fstitle{Lineare Algebra}

\section{Grundlagen}
TODO: Modulo und $\mathbb{Z_n}$

\subsection{Mengenlehre}
\begin{sectionbox}
    \subsubsection{Für alle Mengen A,B,C gilt:}
    \begin{enumerate}\itemsep-1pt
        \item $\emptyset \subset B $
        \item $A \setminus (B \cup C) = (A \setminus B) \cap (A \setminus C)$
        \item $(A \cap B) \cap C = A \cap (B \cap C)$\\
              $(A \cup B) \cup C = A \cup (B \cup C)$
        \item $A \cap (B \cup C) = (A \cap B) \cup (A \cap C) \\
                  A \cup (B \cap C) = (A \cup B) \cap (A \cup C)$
        \item $A \triangle B = (A \cup B) \setminus (A \cap B)$\\
              symmetr. Differenz zweier Mengen: Alle Elemente, die entweder in A oder B enthalten sind.
        \item $A \times B=\{(a_1, b_1), ..., (a_1, b_n), (a_2, b_1), ...\}$ \\
              direktes Produkt zweier Mengen, ordnet jedem Element aus $A$ jedes Element aus $B$ zu (Tupel)
    \end{enumerate}
\end{sectionbox}

\subsection{Abbildungen}
\begin{sectionbox}
    $f: A \ra B$\\
    \textbf{Surjektivität}\\
    $\forall y \in B \exists x\in A: f(x)=y$. Es wird auf alle Werte der Wertemenge abgebildet\\
    Beweisstruktur:\\
    \vspace{-2mm}
    \begin{itemize}\itemsep1pt
        \item Sei $y\in B$, finde $x \in A$ mit $f(x)=y$
        \item Durch Einsetzen ergibt sich $y=f(x)$, kann man dies nach $x$ auflösen, so ist $f$ surjektiv. \\
              Ist $B=K^n$, z.B. $n=2$, so wählt man $(a, b) \in K^2, f(x, y)=(a,b)$ und löst das lin. Gleichungssystem nach $x$ und $y$ auf
    \end{itemize}
    \colorbox{yellow}{$f$ surjektiv \equival $\Bild f = B$}

    \textbf{Injektivität}\\
    $f(x_1)=f(x_2) \Rightarrow x_1=x_2$. Zwei verschiedene Elemente der Definitionsmenge bilden nicht auf gleiche Elemente der Wertemenge ab.\\
    Beweisstruktur:\\
    \vspace{-2mm}
    \begin{itemize}
        \item $x_1, x_2 \in A$ mit $x_1\ne x_2$
        \item Gibt es eine Kombination von $x_1, x_2$, sodass $f(x_1)=f(x_2)$, so ist $f$ nicht injektiv\\
              Ist $A=K^n$, z.B. $n=2$, so wählt man $(x_1, y_1), (x_2, y_2) \in K^2$ und vergleicht $f(x_1, y_1)=f(x_2,y_2)$. Hier erhält man zwei Gleichungen, setzt man die eine in die andere ein, so muss $x_1=x_2$ bzw. $y_1=y_2$ rauskommen.
    \end{itemize}
    \colorbox{yellow}{$f$ injektiv \equival $\Kern f = \{0\}$}\\
    \colorbox{yellow}{$f$ injektiv \equival $f$ streng monoton steigend}
\end{sectionbox}

\begin{sectionbox}
    \subsubsection{Schnitt-/Vereinigungsbeweis von (Ur-)Bildern}
    \begin{itemize}
        \item $f^{-1}(N_1 \cup N_2) = f^{-1}(N_1) \cup f^{-1}(N_2)$\\
              \vspace{-2mm}
              \begin{enumerate}
                  \item Sei $x\in f^{-1}(N_1 \cup N_2) \equival f(x) \in N_1 \cup N_2$
                  \item Beweis weiterführen, bis man auf $x \in f^{-1}(N_1) \cup f^{-1}(N_2)$ kommt
              \end{enumerate}
    \end{itemize}
    \begin{itemize}
        \item $f(M_1 \cup M_2) = f(M_1) \cup f(M_2)$\\
              \vspace{-2mm}
              \begin{enumerate}
                  \item Sei $y \in f(M_1 \cup M_2)$ \equival $x\in M_1 \cup M_2$
                  \item Beweis weiterführen, bis man auf $y \in f(M_1) \cup f(M_2)$ kommt
              \end{enumerate}
    \end{itemize}
\end{sectionbox}

\section{Algebraische Strukturen}
\subsection{Gruppen}
Menge mit einer Verknüpfung $\circ$ (z.B. $+$, $\cdot$) $\Rightarrow (M, \circ)$
\begin{sectionbox}
    \subsubsection{Halbgruppe}
    Falls für alle $a,b,c \in M$ das Assoziativgesetz gilt:\\
    $(ab)c \overset{!}{=} a(bc)$, hierbei beschreibt $ab$ die Verknüpfung von $a$ und $b$ ($\circ$)
    \subsubsection{Monoid}
    Halbgruppe mit neutralem Element $e$: $ae=ea=a$
    \subsubsection{abelsche/kommutative (Halb)-Gruppe}
    (Halb-)Gruppe, wobei das Kommutativgesetz gilt: $ab=ba$
\end{sectionbox}

\begin{sectionbox}
    \subsubsection{Gruppe}
    Eine Halbgruppe $(G, \circ)$ heißt Gruppe mit neutralem Element $e$, falls:
    \begin{enumerate}
        \item Die Gruppe abgeschlossen ist: $\forall a,b \in G:a \circ b \in G$
        \item  es zu jedem $a \in G$ ein $b\in G$ gibt mit $ab=ba=e$.\\
              $b$ ist das Inverse von $a$ und ist eindeutig. Man schreibt $b=a^{-1}$
    \end{enumerate}
    \textbf{symmetrische Gruppe $S_n$:} Gruppe der Permutationen:\\
    $(\overset{\textcolor{green}{1}}{1}, \overset{\textcolor{orange}{2}}{3}, \overset{\textcolor{green}{3}}{2})(\overset{\textcolor{red}{1}}{2}, \overset{\textcolor{green}{2}}{3}, \overset{\textcolor{green}{3}}{1})=(3,2,1)$.
    Von rechts nach links gelesen:
    \begin{enumerate}
        \item Die Stelle $\textcolor{red}{1}$ wird auf die $2$ abgebildet, die Stelle $\textcolor{orange}{2}$ wird auf die $3$ abgebildet. 1. Element des Ergebnisses, usw...
    \end{enumerate}
\end{sectionbox}

\begin{sectionbox}
    \subsubsection{Untergruppe}
    Gruppe $(U, \circ)$, wobei $U \subset G$. Dann ist $U$ Untergruppe von $G$ \\
    \textbf{Axiome zum Beweis von Untergruppe:}
    \begin{enumerate}
        \item $e \in U$
        \item $v,w \in U \Rightarrow v\circ w \in U$
        \item $v \in U \Rightarrow v^{-1} \in U$
    \end{enumerate}
\end{sectionbox}

\begin{sectionbox}
    \subsection{Ring}
    Menge $R$ mit zwei Verknüpfungen $(R, +, \cdot)$, falls gilt:
    \begin{enumerate}
        \item $(R, +)$ kommutative Gruppe mit neutralem Element $\colorbox{yellow}{e}$
        \item $(R, \cdot)$ Halbgruppe
        \item $\forall a,b,c \in R$ gilt $a(b+c)=ab+ac$ und $(b+c)a=ba+ca$ (Distributivgesetz)
    \end{enumerate}
\end{sectionbox}

\begin{sectionbox}
    \subsection{Körper}
    \begin{enumerate}
        \item $(R, +, \cdot)$ ist ein Ring
        \item $(R\setminus \{\colorbox{yellow}{e}\text{ der kommutativen Gruppe des Rings}\}, \cdot)$ ist kommutative Gruppe
    \end{enumerate}

    \textbf{Wichtige Beispiele}
    \begin{itemize}
        \item $\mathbb{Q}$ und $\R$ sind Körper
        \item $(\mathbb{Z}_p, \oplus_p, \odot_p)$ ist für jede Primzahl $p$ ein Körper
    \end{itemize}
\end{sectionbox}

\begin{sectionbox}
    \subsection{Vektorraum}
    Eine Menge V ist K-Vektorraum (VR über Körper K) mit
    \begin{enumerate}
        \item Vektoraddition: $V \times V\rightarrow V, (v,w) \mapsto v + w$
        \item Skalarmultiplikation: $K\times V \rightarrow V, (\lambda, v)\mapsto \lambda v$
    \end{enumerate}
    wobei $\forall \lambda, \mu \in K$ und $\forall v,w \in V$ gilt:
    \begin{enumerate}
        \item $(\lambda \mu)v=\lambda(\mu v)$
        \item $\lambda(v+w)=\lambda v + \lambda w$ und $(\lambda + \mu)v=\lambda v + \mu v$
        \item $1v = v$
    \end{enumerate}
\end{sectionbox}

\begin{sectionbox}
    \subsubsection{Untervektorraum}
    $U$ ist Untervektorraum von $V$, wenn $U \subset V$, bzw. $\forall v,w \in U, \lambda \in K$
    \begin{enumerate}
        \item $v+w \in U$
        \item $\lambda v \in U$
        \item $U \neq \emptyset$ bzw. $e \in U$ ($\colorbox{yellow}{e}$ der kommut. Gruppe des Ringes)
    \end{enumerate}

    \subsubsection{direkte Summe zweier UVRs}
    Zwei UVRs $U_1, U_2$ heißen direkte Summe $U_1 \oplus U_2$, wenn $U_1 \cap U_2= \{0\}$. D.h. zwei UVRs treffen sich nur im Ursprung (zwei Geraden (UVR vom $R^3$) schneiden sich nur im Ursprung)\\
    In der Summe $U_1+U_2=\{u_1 + u_2 \vert u_1 \in U_1, u_2 \in U_2\}$ ist die Darstellung $u=u_1 + u_2$ eindeutig. Es gibt nur eine mögliche Linearkombination! (Aus ) Z.B.\\
    $\{(x, 0) | x \in \mathbb{R}\} \oplus \{(0, y) | x \in \mathbb{R}\} = \mathbb{R}^2$, da $(x,y) = (x,0) + (0,y)$

    \subsubsection{Neue UVRs aus Alten}
    \begin{itemize}
        \item $U_1 \cap U_2$ ist UVR
        \item $U_1 + U_2$ ist UVR
        \item $\Phi : V \rightarrow W$ linear mit $U \subset V, T \subset W$: \\
              $\Phi(U)=\{\Phi(u) | u \in U\}$ UVR von $W$\\
              $\Phi(T)=\{v \in V | \Phi(v) \in T\}$ UVR von $V$
    \end{itemize}
\end{sectionbox}

\begin{sectionbox}
    \subsection{Spann}
    Menge aller Vektoren, die aus Linearkombination der Argumente (hier $v_1, v_2$) hervorgehen,\\
    z.B.: $\spann(v_1, v_2)=\{av_1 + bv_2 | a,b \in \mathbb{R}\}$
    \begin{itemize}
        \item $\spann M$ ist UVR von $V$, wenn $M \subset V$
        \item Sind $U_1$ und $U_2$ UVR von $V$: $\spann(U_1 \cup U_2) = U_1 + U_2$
        \item $\spann \emptyset = \{0\}$
        \item $\Phi : V \rightarrow W$ linear: $\spann(\Phi(V))=\Phi(\spann V)$
    \end{itemize}
\end{sectionbox}

\section{Lineare Abbildungen}
Seien $V,W$ $K$-Vektorräume. Eine Abbildung $\Phi:V\rightarrow W$ heißt \textbf{Vektorraumhomomorphismus} oder \textbf{linear}, falls $\forall v,w \in V$ und $\forall \lambda \in K$ gilt:
\begin{itemize}
    \item $\Phi(0)=0$
    \item $\Phi(\lambda v + w) = \lambda\Phi(v) + \Phi(w)$
\end{itemize}
$L(V,W):=\{\Phi: V \rightarrow W | \Phi \text{ linear} \}$

\begin{sectionbox}
    \textbf{Endomorphismen: }VR-Homomorphismen von $V$ auf sich selbst ($\Phi:V\rightarrow V$) \\
    \textbf{Isomorphismen: }Bijektive VR-Homomorphismen\\
    \textbf{Automorphismen: }Bijektive Endomorphismen\\
\end{sectionbox}

\begin{sectionbox}
    \subsection{Kern}
    Sei $\Phi : V \rightarrow W$ eine lineare Abbildung, der $\Kern \Phi$ ist die Menge aller Elemente aus $V$, die auf die $0$ abgebildet werden.\\
    $\Kern \Phi := \{v \in V | \Phi(v) = 0\}=\Phi^{-1}(\{0\})$
\end{sectionbox}

\begin{sectionbox}
    \subsection{Bild}
    Sei $\Phi : V \rightarrow W$ eine lineare Abbildung, das $\Bild \Phi$ ist die Menge aller Elemente aus $W$, auf die abgebildet wird.\\
    $\Bild \Phi := \{\Phi(v) | v \in V\}$
\end{sectionbox}

\begin{sectionbox}
    \subsection{Dimension}
    Menge an lin. unabh. Vektoren eines VRs ($f: V\rightarrow W$ lin.)\\
    \textbf{Rang:} $\Rang f := \dim(\Bild f)$

    \textbf{Dimensionssatz:} $\dim V = \dim(\Kern f) + \dim(\Bild f)$
    \begin{itemize}
        \item $\dim V = 0 \equival V=\{0\}$
    \end{itemize}
\end{sectionbox}

\begin{sectionbox}
    \subsection{Vektoraumisomorphismus}
    Zwei VRs sind isomorph $V \tilde{=} W$ genau dann, wenn
    \begin{itemize}
        \item Man eine bijektive lin. Abb. finden kann, die eine Basis von $V$ auf eine Basis von $W$ abbildet
        \item $\dim V = \dim W$
    \end{itemize}
    \textbf{Koordinatensystem:} Isomorphismus $\Phi:K^n \rightarrow V$
\end{sectionbox}

CONTINUE: Determinanten und Bilinearform

\subsection{Basiswechsel}
\begin{sectionbox}
    \subsubsection{Basiswechselmatrix allg.}
    $[v]_{B_{neu}}=P_{B_{neu} \leftarrow B_{alt}}[v]_{B_{alt}}$\\
    Basiswechselmatrix finden, die $B_{alt}$ zu $B_{neu}$ abbildet:
    \begin{enumerate}
        \item Nimm den $j$-ten ($v_j$) Basisvektor aus $B_{alt}$
        \item Schreibe ihn in der neuen Basis mithilfe Linearkombination der neuen Basisvektoren\\
              $v_j=a_1w_1+...+a_nw_n$
        \item Die $j$-te Spalte von $P_{B_{neu} \leftarrow B_{alt}}$ ist der Vektor der Koeffizienten $(a_1, ..., a_n)^T$
    \end{enumerate}
\end{sectionbox}

\begin{sectionbox}
    \subsubsection{Basiswechselmatrix $R^n$}
    Basiswechselmatrix von $B_{alt}$ zu $B_{neu}$:
    \begin{enumerate}
        \item Basisvektoren von $B_{alt}$ und $B_{neu}$ als Matrix nebeneinander schreiben
        \item $P_{B_{neu} \leftarrow B_{alt}}=B_{neu}^{-1}B_{alt}$\\
              $P_{B_{alt} \leftarrow B_{neu}}=B_{alt}^{-1}B_{neu}$
    \end{enumerate}
\end{sectionbox}

\begin{sectionbox}
    \subsection{Darstellungsmatrix $D$}
    Beschreibt die Bilder der Basisvektoren einer lin. Abb.\\

    \subsubsection{Basiswechsel mit Basiswechselmatrix}
    Basiswechsel im Definitionsraum mit neuer Basis $B_{neu}$:\\
    $D_{B_{neu}, B}=D_{B_{alt},B} \times P_{B_{alt} \leftarrow B_{neu}}$

    Basiswechsel im Bildraum mit neuer Basis $B_{neu}$:\\
    $D_{B, B_{neu}}=P_{B_{neu} \leftarrow B_{alt}} \times D_{B, B_{alt}}$

    \subsubsection{Spezialfälle}
    Basiswechsel im Definitionsraum mit Standard-Basis im Bildraum:\\
    Bilder der neuen Basisvektoren sind die Spalten der Darstellungsmatrix\\

    Basiswechsel im Bildraum mit Standard-Basis im Definitionsraum:\\
    Die Bilder der Standard-Basisvektoren ausgedrückt in der neuen Basis $\{b_1, ..., b_n\}$ bilden die Spalten der Darstellungsmatrix:
    $f(e_n)\overset{!}{=}a_1b_1 + ... + a_nb_n$.
    Die Koeffizienten bilden die Spalten der Darstellungsmatrix.
\end{sectionbox}

\subsection{Determinanten}
\begin{sectionbox}
    $f:V \rightarrow V,\quad \omega$ Determinantenform: \\
    $(\det f)\omega(v_1,...,v_n)=\omega(f(v_1), ..., f(v_n))$
    \subsubsection{Eigenschaften}
    \begin{itemize}
        \item Vertauscht man zwei Spalten/Zeilen von $A \rightarrow A' \Longrightarrow \det A = -\det A'$
        \item Skaliert man $n$ Spalten mit $\lambda$, $A \rightarrow A' \equival \det A = \lambda^n \det A'$
        \item Addiert man zwei Spalten aufeinander: $\det A$ bleibt gleich
        \item $\det(f \circ g)=\det (AB)=\det f \det g = \det A \det B$
        \item $\det A^{-1}=\frac{1}{\det A}$
        \item $\det A = \det A^T$
    \end{itemize}
\end{sectionbox}

\subsubsection{Laplace-Formel}
Vorzeichenwechsel beispielhaft erklärt für $K^{3\times 3}$\\
$\begin{array}{c|c|c|c}
        \text{Term} & \text{Spalte} & \text{sign} & \text{\# Vertauschungen} \\ \hline
        aei         & (1,2,3)       & +           & 0                        \\
        afg         & (1,3,2)       & -           & 1                        \\
        bfg&(2,3,1)&-&2
    \end{array}$

\subsection{euklidische Vektorräume}
\textbf{Cauchy-Schwarz-Ungl.:} $|\langle v,w \rangle| \le ||v|| \: ||w||$

\begin{sectionbox}
    Sei $(V, \langle \cdot, \cdot \rangle)$ euklid. VR. Dann gilt $\forall v,w \in V$:
    \begin{enumerate}
        \item \textbf{Polarisation:} $\langle v,w \rangle = \frac{1}{4}(||v+w||^2-||v-w||^2)$
        \item \textbf{Parallelogrammgleichung:}\\
        $2(||v||^2+||w||^2)=||v+w||^2+||v-w||^2$
    \end{enumerate}

    Für eine bel. Norm gilt: Die Parallelogrammgleichung gilt \equival \: Die Norm kommt von einem Skalarprodukt. Dann erhält man das Skalarprodukt durch Polarisation.
\end{sectionbox}

\begin{sectionbox}
    \subsubsection{Skalarprodukt}
\begin{itemize}
    \item bilinear: $\langle a,b+c\rangle = \langle a,b\rangle + \langle a,c \rangle$
    \item symmetr.: $\langle a,b \rangle = \langle b,a \rangle$
    \item pos. definit: $\langle v,v\rangle > 0$ oder $\langle v,v\rangle = 0 \Rightarrow v = 0$
\end{itemize}
\end{sectionbox}

\begin{sectionbox}
    \subsubsection{Norm}
    Abbildung $||\cdot||:V \rightarrow\mathbb R$ mit den Eigenschaften:
    \begin{itemize}
        \item $||\lambda v||=|\lambda| \: ||v|| \quad \forall \lambda \in \mathbb R, v \in V$
        \item $||a|| > 0$ oder $||a|| = 0 \Rightarrow a = 0$
        \item $\forall v,w \in V$ gilt die Dreiecksungleichung: \\
        $||v + w|| \le ||v|| + ||w||$
    \end{itemize}
\end{sectionbox}

\begin{sectionbox}
    
\end{sectionbox}

\section{Matrix-Lookup-Table}
\begin{sectionbox}
    \subsection{Rechenregeln}
    \begin{itemize}
        \item $A \in \mathbb R^{m \times n}$, $B \in \mathbb R^{n\times k} \rightarrow AB$ definiert: $AB \in \mathbb R^{m \times k}$
        \item $\det A = a_{11}a_{22}-a_{12}a_{21}$, $A \in \mathbb R^{2\times 2}$
        \item $\det A = a_{11}(a_{22}a_{33}-a_{23}a_{32})-a_{12}(a_{21}a_{33}-a_{23}a_{31})+a_{13}(a_{21}a_{32}-a_{22}a_{31})$
    \end{itemize}
\end{sectionbox}

\begin{sectionbox}
    \subsection{Standard-Matrix}
    \begin{itemize}
        \item Menge an lin. unabh. Spalten/Zeilen von A: $\Rang A$
        \item $\dim(\Kern A)= \text{Spaltenanzahl} - \Rang A$
        \item $f: V \rightarrow W$: $\dim V=$ Spaltenanzahl, $\dim W=$ Zeilenanzahl
        \item $\Rang(AB) \le \min(\Rang A, \Rang B)$
    \end{itemize}

    \subsubsection{Invertierbarkeit $A \in \operatorname{Mat}(n,n,K)$}
    \begin{itemize}
        \item $A$ invertb. $\equival \Rang A = n \equival \dim(\Kern A) = 0 \rightarrow A$ besitzt keine Nullzeile/-spalte
        \item $A$ invertierbar $\equival A$ bijektiv
        \item $A$ invertierbar $\equival \det A \ne 0$
        \item $A$ Nullzeile/-spalte $\equival A$ nicht invertb.
    \end{itemize}

    \subsubsection{Transposition}
    \begin{itemize}
        \item $\Rang A=\Rang A^T$ (Zeilenrang = Spaltenrang)
        \item symmetr. Matrix: $A^T=A$: $a_{ji}=a_{ij}$
        \item schiefsymmetr. Matrix: $A^T=-A^T$: $a_{ji}=-a_{ij}$
        \item $(A^T)^{-1}=(A^{-1})^T$
    \end{itemize}
\end{sectionbox}

\begin{sectionbox}
    \subsection{Gauß}
    \begin{itemize}
        \item Anzahl der lin. unabh. Nicht-Nullzeilen $\equival \Rang A$
    \end{itemize}

    \subsubsection{Lösbarkeit der erw. Matrix $[A|b]$}
    \begin{itemize}
        \item Unlösbar: Es existiert Nullzeile links und ein Nicht-Null Eintrag rechts
        \item Eindeutig lösbar: $\Rang A =$ Anzahl der Variablen und $b \in \Bild A$
        \item Unendl. viele Lösungen: $\Rang A <$ Anzahl der Variablen $\rightarrow$ Nullzeile rechts und Nulleintrag links
        \item Homog. System $b=0$: Lösungsmenge ist VR - der $\Kern A \Rightarrow \dim L=\dim V - \Rang A$
        \item Inhomog. System: Lösungsmenge ist Partikularlösung + Kern (affiner Unterraum)
    \end{itemize}
\end{sectionbox}

\label{LastPage}
\end{document}