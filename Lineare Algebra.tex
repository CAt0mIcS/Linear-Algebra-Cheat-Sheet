\documentclass[german, fourcolumn, 8pt]{latex4ei/latex4ei_sheet}
% LastPage
\usepackage{lastpage}
\usepackage{amsmath}

% Allow hyperlinks
	\RequirePackage[pagebackref=true,pdfpagelabels]{hyperref}
	
% Colors
	\RequirePackage{latex4ei/latex4ei_colors}
	\colorlet{col_link}{tum_blue_dark}
	\hypersetup{
	colorlinks=true,
	linkcolor=col_link,
	urlcolor=col_link,
	citecolor=col_link,
}

\newcommand{\equival}[0]{\ensuremath{\Longleftrightarrow}}
\newcommand{\spann}{\ensuremath{\text{spann}}}
\newcommand{\Kern}{\ensuremath{\text{Kern }}}
\newcommand{\Bild}{\ensuremath{\text{Bild }}}
	
\begin{document}

\titlespacing*{\subsubsection} {0pt}{0.5ex}{0.0ex}
\titlespacing*{\subsection} {0pt}{1.0ex}{1.0ex}
\titleformat{\subsubsection}
{\normalfont\normalsize\bfseries}{\thesubsubsection}{.5em}{}


% Aufteilung in Spalten
\vspace{-4mm}
\fstitle{Lineare Algebra}

\section{Grundlagen}
\textbf{Dreiecksungleichung}\\
\begin{math}\begin{array}{l}
        \abs{x + y} \le \abs{x} + \abs{y} \\
        \abs{\abs{x}- \abs{y}} \le \abs{x-y}
    \end{array}\end{math} \\
TODO: Modulo und $\mathbb{Z_n}$

\subsection{Mengenlehre}
\begin{sectionbox}
    \subsubsection{Für alle Mengen A,B,C gilt:}
    \begin{enumerate}\itemsep-1pt
        \item $\emptyset \subset B $
        \item $A \setminus (B \cup C) = (A \setminus B) \cap (A \setminus C)$
        \item $(A \cap B) \cap C = A \cap (B \cap C)$\\
                $(A \cup B) \cup C = A \cup (B \cup C)$
        \item $A \cap (B \cup C) = (A \cap B) \cup (A \cap C) \\
                    A \cup (B \cap C) = (A \cup B) \cap (A \cup C)$
        \item $A \triangle B = (A \cup B) \setminus (A \cap B)$\\
                symmetr. Differenz zweier Mengen: Alle Elemente, die entweder in A oder B enthalten sind.
    \end{enumerate}
\end{sectionbox}

\subsection{Abbildungen}
\begin{sectionbox}
    $f: A \ra B$\\
    \textbf{Surjektivität}\\
    $\forall y \in B \exists x\in A: f(x)=y$. Es wird auf alle Werte der Wertemenge abgebildet\\
    Beweisstruktur:\\
    \vspace{-2mm}
    \begin{itemize}\itemsep1pt
        \item Sei $y\in B$, finde $x \in A$ mit $f(x)=y$
        \item Durch Einsetzen ergibt sich $y=f(x)$, kann man dies nach $x$ auflösen, so ist $f$ surjektiv. \\
                Ist $B=K^n$, z.B. $n=2$, so wählt man $(a, b) \in K^2, f(x, y)=(a,b)$ und löst das lin. Gleichungssystem nach $x$ und $y$ auf
    \end{itemize}

    \textbf{Injektivität}\\
    $f(x_1)=f(x_2) \Rightarrow x_1=x_2$. Zwei verschiedene Elemente der Definitionsmenge bilden nicht auf gleiche Elemente der Wertemenge ab.\\
    Beweisstruktur:\\
    \vspace{-2mm}
    \begin{itemize}
        \item $x_1, x_2 \in A$ mit $x_1\ne x_2$
        \item Gibt es eine Kombination von $x_1, x_2$, sodass $f(x_1)=f(x_2)$, so ist $f$ nicht injektiv\\
                Ist $A=K^n$, z.B. $n=2$, so wählt man $(x_1, y_1), (x_2, y_2) \in K^2$ und vergleicht $f(x_1, y_1)=f(x_2,y_2)$. Hier erhält man zwei Gleichungen, setzt man die eine in die andere ein, so muss $x_1=x_2$ bzw. $y_1=y_2$ rauskommen.
    \end{itemize}
    \colorbox{yellow}{$f$ injektiv \equival $\Kern f = \{0\}$}\\
    \colorbox{yellow}{$f$ injektiv \equival $f$ streng monoton steigend}
\end{sectionbox}

\begin{sectionbox}
    \subsubsection{Schnitt-/Vereinigungsbeweis von (Ur-)Bildern}
    \begin{itemize}
        \item $f^{-1}(N_1 \cup N_2) = f^{-1}(N_1) \cup f^{-1}(N_2)$\\
                \vspace{-2mm}
                \begin{enumerate}
                    \item Sei $x\in f^{-1}(N_1 \cup N_2) \equival f(x) \in N_1 \cup N_2$
                    \item Beweis weiterführen, bis man auf $x \in f^{-1}(N_1) \cup f^{-1}(N_2)$ kommt
                \end{enumerate}
    \end{itemize}
    \begin{itemize}
        \item $f(M_1 \cup M_2) = f(M_1) \cup f(M_2)$\\
                \vspace{-2mm}
                \begin{enumerate}
                    \item Sei $y \in f(M_1 \cup M_2)$ \equival $x\in M_1 \cup M_2$
                    \item Beweis weiterführen, bis man auf $y \in f(M_1) \cup f(M_2)$ kommt
                \end{enumerate}
    \end{itemize}
\end{sectionbox}

\section{Algebraische Strukturen}
\subsection{Gruppen}
Menge mit einer Verknüpfung $\circ$ (z.B. $+$, $\cdot$) $\Rightarrow (M, \circ)$
\begin{sectionbox}
    \subsubsection{Halbgruppe}
    Falls für alle $a,b,c \in M$ das Assoziativgesetz gilt:\\
    $(ab)c \overset{!}{=} a(bc)$, hierbei beschreibt $ab$ die Verknüpfung von $a$ und $b$ ($\circ$)
    \subsubsection{Monoid}
    Halbgruppe mit neutralem Element $e$: $ae=ea=a$
    \subsubsection{abelsche/kommutative (Halb)-Gruppe}
    (Halb-)Gruppe, wobei das Kommutativgesetz gilt: $ab=ba$
\end{sectionbox}

\begin{sectionbox}
    \subsubsection{Gruppe}
    Eine Halbgruppe $(G, \circ)$ heißt Gruppe mit neutralem Element $e$, falls:
    \begin{enumerate}
        \item Die Gruppe abgeschlossen ist: $\forall a,b \in G:a \circ b \in G$
        \item  es zu jedem $a \in G$ ein $b\in G$ gibt mit $ab=ba=e$.\\
                $b$ ist das Inverse von $a$ und ist eindeutig. Man schreibt $b=a^{-1}$
    \end{enumerate}
    \textbf{symmetrische Gruppe $S_n$:} Gruppe der Permutationen:\\
    $(\overset{\textcolor{green}{1}}{1}, \overset{\textcolor{green}{2}}{3}, \overset{\textcolor{green}{3}}{2})(\overset{\textcolor{green}{1}}{2}, \overset{\textcolor{green}{2}}{3}, \overset{\textcolor{green}{3}}{1})=(3,2,1)$.
    Von rechts nach links gelesen:
    \begin{enumerate}
        \item Die Stelle $\textcolor{green}{1}$ wird auf die $2$ abgebildet, die Stelle $\textcolor{green}{2}$ wird auf die $3$ abgebildet. 1. Element des Ergebnisses, usw...
    \end{enumerate}
\end{sectionbox}

\begin{sectionbox}
    \subsubsection{Untergruppe}
    Gruppe $(U \circ)$, wobei $U \subset G$. Dann ist $U$ Untergruppe von $G$ \\
    \textbf{Axiome zum Beweis von Untergruppe:}
    \begin{enumerate}
        \item $e \in U$
        \item $v,w \in U \Rightarrow v\circ w \in U$
        \item $v \in U \Rightarrow v^{-1} \in U$
    \end{enumerate}
\end{sectionbox}

\begin{sectionbox}
    \subsection{Ring}
    Menge $R$ mit zwei Verknüpfungen $(R, +, \cdot)$, falls gilt:
    \begin{enumerate}
        \item $(R, +)$ kommutative Gruppe mit neutralem Element $\colorbox{yellow}{e}$
        \item $(R, \cdot)$ Halbgruppe
        \item $\forall a,b,c \in R$ gilt $a(b+c)=ab+ac$ und $(b+c)a=ba+ca$ (Distributivgesetz)
    \end{enumerate}
\end{sectionbox}

\begin{sectionbox}
    \subsection{Körper}
    \begin{enumerate}
        \item $(R, +, \cdot)$ ist ein Ring
        \item $(R\setminus \{\colorbox{yellow}{e}\text{ der kommutativen Gruppe des Rings}\}, \cdot)$ ist kommutative Gruppe
    \end{enumerate}

    \textbf{Wichtige Beispiele}
    \begin{itemize}
        \item $\mathbb{Q}$ und $\R$ sind Körper
        \item $(\mathbb{Z}_p, \oplus_p, \odot_p)$ ist für jede Primzahl $p$ ein Körper
    \end{itemize}
\end{sectionbox}

\begin{sectionbox}
    \subsection{Vektorraum}
    Eine Menge V ist K-Vektorraum (VR über Körper K) mit
    \begin{enumerate}
        \item Vektoraddition: $V \times V\rightarrow V, (v,w) \mapsto v + w$
        \item Skalarmultiplikation: $K\times V \rightarrow V, (\lambda, v)\mapsto \lambda v$
    \end{enumerate}
    wobei $\forall \lambda, \mu \in K$ und $v,w \in V$ gilt:
    \begin{enumerate}
        \item $(\lambda \mu)v=\lambda(\mu v)$
        \item $\lambda(v+w)=\lambda v + \lambda w$ und $(\lambda + \mu)v=\lambda v + \mu v$
        \item $1v = v$
    \end{enumerate}
\end{sectionbox}

\begin{sectionbox}
    \subsubsection{Untervektorraum}
    $U$ ist Untervektorraum von $V$, wenn $U \subset V$, bzw. $\forall v,w \in U, \lambda \in K$
    \begin{enumerate}
        \item $v+w \in U$
        \item $\lambda v \in U$
        \item $U \neq \emptyset$ bzw. $e \in U$ (TODO: Welches $e$)
    \end{enumerate}
\end{sectionbox}

TODO: Neue Vektorräume aus Alten

\begin{sectionbox}
    \subsection{Spann}
    $\spann(M)$ ist die Menge an TODO\\
    $\spann(M)$ ist UVR von $V$, wenn $M \subset V$\\
    Sind $U_1$ und $U_2$ UVR von $V$: $\spann(U_1 \cup U_2) = U_1 + U_2$
\end{sectionbox}

\section{Lineare Abbildungen}
Seien $V,W$ $K$-Vektorräume. Eine Abbildung $\Phi:V\rightarrow W$ heißt \textbf{Vektorraumhomomorphismus} oder \textbf{linear}, falls $\forall v,w \in V$ und $\forall \lambda \in K$ gilt:
\begin{itemize}
    \item $\Phi(0)=0$
    \item $\Phi(\lambda v + w) = \lambda\Phi(v) + \Phi(w)$
\end{itemize}

\begin{sectionbox}
    \textbf{Endomorphismen: }Vektorraumhomomorphismen von $V$ auf sich selbst \\($\Phi:V\rightarrow V$) \\
    \textbf{Isomorphismen: }Bijektive Vektorraumhomomorphismen\\
    \textbf{Automorphismen: }Bijektive Endomorphismen\\
    $L(V,W):=\{\Phi: V \rightarrow W | \Phi \text{ linear} \}$
\end{sectionbox}

\begin{sectionbox}
    \subsection{Kern}
    Sei $\Phi : V \rightarrow W$ eine lineare Abbildung, der $\Kern \Phi$ ist die Menge aller Elemente aus $V$, die auf die $0$ abgebildet werden.\\
    $\Kern \Phi := \{v \in V | \Phi(v) = 0\}=\Phi^{-1}(\{0\})$
\end{sectionbox}

\begin{sectionbox}
    \subsection{Bild}
    Sei $\Phi : V \rightarrow W$ eine lineare Abbildung, das $\Bild \Phi$ ist die Menge aller Elemente aus $W$, auf die abgebildet wird.\\
    $\Bild \Phi := \{\Phi(v) | v \in V\}$
\end{sectionbox}

CONTINUE HERE: %https://onedrive.live.com/view.aspx?resid=E1A89B133F48E672%21s2c73b2c00ef94b319fd70d82288bec82&id=documents&wd=target%28Linearkombinationen.one%7C9A4C3841-EE0C-4AFA-8706-0A8A255C61B2%2FLinearkombinationen%2C%20lineare%20Unabh%C3%A4ngigkeit%2C%20Basen%7C16CA8DB9-21F8-4DA5-8D69-E0F33835733F%2F%29&wdpartid={DFD224D4-B968-4773-AA0B-89CC2749C977}{1}&wdsectionfileid=E1A89B133F48E672!s2b27107b91f74213818269ca24ba0ee6onenote:https://d.docs.live.net/E1A89B133F48E672/Documents/School/EEIT%20TUM/1st%20Semester/Lineare%20Algebra/Linearkombinationen.one#Linearkombinationen,%20lineare%20Unabhängigkeit,%20Basen&section-id={9A4C3841-EE0C-4AFA-8706-0A8A255C61B2}&page-id={16CA8DB9-21F8-4DA5-8D69-E0F33835733F}&object-id={B9AEBEE8-B2ED-485C-912A-32F91118B3EB}&10

\label{LastPage}
\end{document}